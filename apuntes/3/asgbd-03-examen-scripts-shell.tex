\input{../common/plantilla-ejercicio.tex}
\usepackage{eurosym}



\renewcommand{\hmwkTitle}{Examen de \textit{scripts} de \textit{shell}}


\usepackage{blindtext}



\begin{document}

% \maketitle

% ----------------------------------------------------------------------------------------
%	TABLE OF CONTENTS
% ----------------------------------------------------------------------------------------

% \setcounter{tocdepth}{1} % Uncomment this line if you don't want subsections listed in the ToC

\primerapagina

\setlength{\parindent}{0em}
\setlength{\parskip}{1em}

\section{Normas del examen}
Es difícil evaluar el manejo de \textit{scripts} sin realizar un examen en el ordenador, pero también es difícil condensar un examen en solo dos horas. Por eso, el examen se plantea como una práctica, que el profesor corregirá en clase.

El desarrollo de este ejercicio será como el de otras prácticas. La única diferencia está en que la nota de esta práctica se tendrá en cuenta en el apartado \textit{exámenes} en vez de en el aparatado \textit{prácticas} al calcular la calificación del trimestre.


\begin{homeworkProblem}[: \textit{Scripts} de inicio y parada de \textbf{Oracle} (2 puntos)]

  Crea dos \textit{scripts} para iniciar y parar \textbf{Oracle} en \path{/home/alumno/scripts/oraclestart.sh} y \path{/home/alumno/scripts/oraclestop.sh}
  
\end{homeworkProblem}

\begin{homeworkProblem}[: Arrancar automáticamente \textbf{Oracle} cuando se inicie el servidor]

  \begin{itemize}
  \item \textbf{Oracle} debe levantarse cuando la máquina se inicie, y apagarse cuando la máquina se cierre.
  \item Oracle se iniciará solo si se indica en el fichero \texttt{/etc/oratab}
  
  \item En el fichero \texttt{/home/alumno/logs/oracle.log} se dejará una traza de cuando se arrancó y se paró la máquina, y si fue necesario arrancar o parar \textit{Oracle}. Por ejemplo:
    \begin{listadotxt}[lst:yoracle.log]{Ejemplo de \texttt{/home/alumno/logs/oracle.log} cuando \textbf{Oracle} se arranca}
      2017-02-10-12:40:00 - Máquina arrancada
      2017-02-10-12:40:01 - Oracle arrancando porque /etc/oratab indica Y
      2017-02-10-12:40:20 - Oracle arrancado
      2017-02-10-12:41:00 - Máquina parando
      2017-02-10-12:41:01 - Oracle parando
      2017-02-10-12:41:20 - Oracle parado
    \end{listadotxt}

    \begin{listadotxt}[lst:noracle.log]{Ejemplo de \texttt{/home/alumno/logs/oracle.log} cuando \textbf{Oracle} no se arranca}
      2017-02-10-12:40:00 - Máquina arrancada
      2017-02-10-12:40:01 - Oracle no se arranca porque /etc/oratab indica N
      2017-02-10-12:41:00 - Máquina parando
      2017-02-10-12:41:01 - Oracle parando
      2017-02-10-12:41:10 - Oracle parado
    \end{listadotxt}

    
  \end{itemize}
  
\end{homeworkProblem}


\begin{homeworkProblem}[: Almacena información periódicamente en la base de datos (2 puntos)]

  El usuario de \textbf{Oracle} \texttt{stats} tendrá una tabla \texttt{DF} para poder guardar la salida del comando \texttt{df -k}. Esta tabla tendrá como columnas:
  \begin{itemize}
  \item \texttt{hora}: Hora de lanzamiento del comando
  \item \texttt{sistema}: Nombre del tipo de sistema de ficheros
  \item \texttt{tamano}: Tamaño en KB del sistema de ficheros
  \item \texttt{usado}: Tamaño usado, en KB
  \item \texttt{montado}: Punto de montaje
  \end{itemize}

  % df -k | awk '{print $3}'

  % while read -r line
  % do
  %     name="$line"
  %     echo "Name read from file - $name"
  %     done < "$filename"

  Programa un \textit{script} para que cada minuto almacene en la tabla \texttt{DF} la información del comando \texttt{df -k}

  \begin{listadotxt}[lst:df-k]{Ejemplo de salida del comando \texttt{df -k}}
Filesystem     1K-blocks      Used Available Use% Mounted on
udev             4002180         0   4002180   0% /dev
tmpfs             804488     19756    784732   3% /run
/dev/sda1      237874840 183034916  42733532  82% /
tmpfs            4022440    437328   3585112  11% /dev/shm
tmpfs               5120         4      5116   1% /run/lock
tmpfs            4022440         0   4022440   0% /sys/fs/cgroup
/dev/sdb5      689521880 595546232  58926896  91% /home/windows
cgmfs                100         0       100   0% /run/cgmanager/fs
tmpfs             804488        88    804400   1% /run/user/1000
  \end{listadotxt}
  

  Pistas para realizar el \textit{script}:
  \begin{itemize}
  \item Pasar de cadena a fecha en \textbf{Oracle}: \enlace{http://www.dba-oracle.com/f\_to\_date.htm}{http://www.dba-oracle.com/f\_to\_date.htm}
    \item El comando \texttt{date} de \textbf{linux} puede también formatearse
  \item Los \textit{heredocs} pueden contener variables: \enlace{http://superuser.com/questions/456615/how-to-pass-variables-to-a-heredoc-in-bash}{http://superuser.com/questions/456615/how-to-pass-variables-to-a-heredoc-in-bash}
  \item Cortar columnas: \enlace{https://www.cyberciti.biz/tips/processing-the-delimited-files-using-cut-and-awk.html}{https://www.cyberciti.biz/tips/processing-the-delimited-files-using-cut-and-awk.html}
  \item Leer líneas: \enlace{http://stackoverflow.com/questions/10929453/read-a-file-line-by-line-assigning-the-value-to-a-variable}{http://stackoverflow.com/questions/10929453/read-a-file-line-by-line-assigning-the-value-to-a-variable}
  \end{itemize}


  
\end{homeworkProblem}


\begin{homeworkProblem}[: Envía un correo periódicamente (2 punto)]

  \begin{itemize}
  \item Programa un \textit{script} para que cada minuto envíe un correo con la información promedio del comando  \texttt{df -k}. Puedes usar como base para la consulta el listado \ref{lst:promedio.sql}.
    
    \item Puedes utilizar
  \item El correo se enviará a \path{alvarogonzalez.profesor@gmail.com}
  \item Indica tu nombre en el asunto del correo  


\begin{listadoshell}[lst:promedio.sql]{Consulta tipo para extraer información promedio}
  select 
    sistema, avg(tamano), avg(usado), montado
  from 
    DF
  group by
    sistema, montado;
\end{listadoshell}
    
\end{itemize}
\end{homeworkProblem}


\section{Instrucciones de entrega}
\begin{itemize}
\item El ejercicio se realizará y entregará de manera individual.
\item El profesor comprobará el funcionamiento del sistema, hasta el dia 14 de Febrero.
  
\end{itemize}
\end{document}
