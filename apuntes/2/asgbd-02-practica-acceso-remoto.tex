\input{../common/plantilla-ejercicio.tex}
\usepackage{eurosym}





\renewcommand{\hmwkTitle}{Conexión remota con \textbf{SQLDeveloper}}


\usepackage{blindtext}

\begin{document}

% \maketitle

% ----------------------------------------------------------------------------------------
%	TABLE OF CONTENTS
% ----------------------------------------------------------------------------------------

% \setcounter{tocdepth}{1} % Uncomment this line if you don't want subsections listed in the ToC

\primerapagina

\setlength{\parindent}{2em}
\setlength{\parskip}{1em}

\section{Objetivo de la práctica}
En esta práctica utilizaremos a la base de datos \textbf{Oracle} como un verdadero servidor, conectándonos desde otros ordenadores. Para ello:
\begin{itemize}
\item Necesitamos poner accesible por red nuestro servidor
\item Crearemos un usuario para cada compañero de clase
\item Dejaremos que esos usuarios puedan ver las tablas del ejemplo de \textbf{multas}, pero solo podrán cambiar un campo de una de las tablas
\end{itemize}

\begin{homeworkProblem}[: Poner Oracle disponible por red]



  Los demás compañeros deben poder acceder a nuestro servidor de \textbf{Oracle}. Para ello
  \begin{enumerate}
  \item La máquina virtual debe ser accedida desde el resto del aula. El tipo de conexión será \textit{bridged}
  \item \textbf{Centos} tiene activado un firewall. Hay que desactivarlo como se indica en
    \begin{itemize}
    \item  \url{https://www.liquidweb.com/kb/how-to-stop-and-disable-firewalld-on-centos-7/}
    \end{itemize}
    
  \item La dirección IP se asigna actualmente por DHCP. Esto es un inconveniente porque puede variar cada día. Es mejor utilizar un nombre, así que instalaremos \textbf{avahi}
    \begin{itemize}
    \item \url{https://en.wikipedia.org/wiki/Avahi\_(software)}
    \item \texttt{sudo yum install avahi avahi-tools}
      
    \item Tu ordenador será accesible con el nombre \texttt{nombre-de-host.local}
    \end{itemize}

    
  \end{enumerate}

  Cuando tengas estos cambios, pide al profesor que compruebe que funcionan.
\end{homeworkProblem}


\begin{homeworkProblem}[: Crear usuarios para tus compañeros]

\end{homeworkProblem}


\begin{homeworkProblem}[: Dar acceso a otros usuarios a un campo de tus tablas]

\end{homeworkProblem}


\section{Instrucciones de entrega}
\begin{itemize}
\item El ejercicio se realizará y entregará de manera individual.
  \begin{itemize}
  \item Solo se admiten trabajos en pareja, si en clase es necesario compartir ordenador.
  \end{itemize}
\item Entrega tu trabajo en formato \textbf{doc}, \textbf{docx}, \textbf{odt} o \textbf{pdf}.
\item Sube el documento a \enlace{http://aulavirtual2.educa.madrid.org/course/category.php?id=2724}{la tarea correspondiente en el aula virtual}
\item Presta atención al plazo de entrega (con fecha y hora).
  
\end{itemize}
\end{document}
